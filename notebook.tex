
% Default to the notebook output style

    


% Inherit from the specified cell style.




    
\documentclass[11pt]{article}

    
    
    \usepackage[T1]{fontenc}
    % Nicer default font (+ math font) than Computer Modern for most use cases
    \usepackage{mathpazo}

    % Basic figure setup, for now with no caption control since it's done
    % automatically by Pandoc (which extracts ![](path) syntax from Markdown).
    \usepackage{graphicx}
    % We will generate all images so they have a width \maxwidth. This means
    % that they will get their normal width if they fit onto the page, but
    % are scaled down if they would overflow the margins.
    \makeatletter
    \def\maxwidth{\ifdim\Gin@nat@width>\linewidth\linewidth
    \else\Gin@nat@width\fi}
    \makeatother
    \let\Oldincludegraphics\includegraphics
    % Set max figure width to be 80% of text width, for now hardcoded.
    \renewcommand{\includegraphics}[1]{\Oldincludegraphics[width=.8\maxwidth]{#1}}
    % Ensure that by default, figures have no caption (until we provide a
    % proper Figure object with a Caption API and a way to capture that
    % in the conversion process - todo).
    \usepackage{caption}
    \DeclareCaptionLabelFormat{nolabel}{}
    \captionsetup{labelformat=nolabel}

    \usepackage{adjustbox} % Used to constrain images to a maximum size 
    \usepackage{xcolor} % Allow colors to be defined
    \usepackage{enumerate} % Needed for markdown enumerations to work
    \usepackage{geometry} % Used to adjust the document margins
    \usepackage{amsmath} % Equations
    \usepackage{amssymb} % Equations
    \usepackage{textcomp} % defines textquotesingle
    % Hack from http://tex.stackexchange.com/a/47451/13684:
    \AtBeginDocument{%
        \def\PYZsq{\textquotesingle}% Upright quotes in Pygmentized code
    }
    \usepackage{upquote} % Upright quotes for verbatim code
    \usepackage{eurosym} % defines \euro
    \usepackage[mathletters]{ucs} % Extended unicode (utf-8) support
    \usepackage[utf8x]{inputenc} % Allow utf-8 characters in the tex document
    \usepackage{fancyvrb} % verbatim replacement that allows latex
    \usepackage{grffile} % extends the file name processing of package graphics 
                         % to support a larger range 
    % The hyperref package gives us a pdf with properly built
    % internal navigation ('pdf bookmarks' for the table of contents,
    % internal cross-reference links, web links for URLs, etc.)
    \usepackage{hyperref}
    \usepackage{longtable} % longtable support required by pandoc >1.10
    \usepackage{booktabs}  % table support for pandoc > 1.12.2
    \usepackage[inline]{enumitem} % IRkernel/repr support (it uses the enumerate* environment)
    \usepackage[normalem]{ulem} % ulem is needed to support strikethroughs (\sout)
                                % normalem makes italics be italics, not underlines
    

    
    
    % Colors for the hyperref package
    \definecolor{urlcolor}{rgb}{0,.145,.698}
    \definecolor{linkcolor}{rgb}{.71,0.21,0.01}
    \definecolor{citecolor}{rgb}{.12,.54,.11}

    % ANSI colors
    \definecolor{ansi-black}{HTML}{3E424D}
    \definecolor{ansi-black-intense}{HTML}{282C36}
    \definecolor{ansi-red}{HTML}{E75C58}
    \definecolor{ansi-red-intense}{HTML}{B22B31}
    \definecolor{ansi-green}{HTML}{00A250}
    \definecolor{ansi-green-intense}{HTML}{007427}
    \definecolor{ansi-yellow}{HTML}{DDB62B}
    \definecolor{ansi-yellow-intense}{HTML}{B27D12}
    \definecolor{ansi-blue}{HTML}{208FFB}
    \definecolor{ansi-blue-intense}{HTML}{0065CA}
    \definecolor{ansi-magenta}{HTML}{D160C4}
    \definecolor{ansi-magenta-intense}{HTML}{A03196}
    \definecolor{ansi-cyan}{HTML}{60C6C8}
    \definecolor{ansi-cyan-intense}{HTML}{258F8F}
    \definecolor{ansi-white}{HTML}{C5C1B4}
    \definecolor{ansi-white-intense}{HTML}{A1A6B2}

    % commands and environments needed by pandoc snippets
    % extracted from the output of `pandoc -s`
    \providecommand{\tightlist}{%
      \setlength{\itemsep}{0pt}\setlength{\parskip}{0pt}}
    \DefineVerbatimEnvironment{Highlighting}{Verbatim}{commandchars=\\\{\}}
    % Add ',fontsize=\small' for more characters per line
    \newenvironment{Shaded}{}{}
    \newcommand{\KeywordTok}[1]{\textcolor[rgb]{0.00,0.44,0.13}{\textbf{{#1}}}}
    \newcommand{\DataTypeTok}[1]{\textcolor[rgb]{0.56,0.13,0.00}{{#1}}}
    \newcommand{\DecValTok}[1]{\textcolor[rgb]{0.25,0.63,0.44}{{#1}}}
    \newcommand{\BaseNTok}[1]{\textcolor[rgb]{0.25,0.63,0.44}{{#1}}}
    \newcommand{\FloatTok}[1]{\textcolor[rgb]{0.25,0.63,0.44}{{#1}}}
    \newcommand{\CharTok}[1]{\textcolor[rgb]{0.25,0.44,0.63}{{#1}}}
    \newcommand{\StringTok}[1]{\textcolor[rgb]{0.25,0.44,0.63}{{#1}}}
    \newcommand{\CommentTok}[1]{\textcolor[rgb]{0.38,0.63,0.69}{\textit{{#1}}}}
    \newcommand{\OtherTok}[1]{\textcolor[rgb]{0.00,0.44,0.13}{{#1}}}
    \newcommand{\AlertTok}[1]{\textcolor[rgb]{1.00,0.00,0.00}{\textbf{{#1}}}}
    \newcommand{\FunctionTok}[1]{\textcolor[rgb]{0.02,0.16,0.49}{{#1}}}
    \newcommand{\RegionMarkerTok}[1]{{#1}}
    \newcommand{\ErrorTok}[1]{\textcolor[rgb]{1.00,0.00,0.00}{\textbf{{#1}}}}
    \newcommand{\NormalTok}[1]{{#1}}
    
    % Additional commands for more recent versions of Pandoc
    \newcommand{\ConstantTok}[1]{\textcolor[rgb]{0.53,0.00,0.00}{{#1}}}
    \newcommand{\SpecialCharTok}[1]{\textcolor[rgb]{0.25,0.44,0.63}{{#1}}}
    \newcommand{\VerbatimStringTok}[1]{\textcolor[rgb]{0.25,0.44,0.63}{{#1}}}
    \newcommand{\SpecialStringTok}[1]{\textcolor[rgb]{0.73,0.40,0.53}{{#1}}}
    \newcommand{\ImportTok}[1]{{#1}}
    \newcommand{\DocumentationTok}[1]{\textcolor[rgb]{0.73,0.13,0.13}{\textit{{#1}}}}
    \newcommand{\AnnotationTok}[1]{\textcolor[rgb]{0.38,0.63,0.69}{\textbf{\textit{{#1}}}}}
    \newcommand{\CommentVarTok}[1]{\textcolor[rgb]{0.38,0.63,0.69}{\textbf{\textit{{#1}}}}}
    \newcommand{\VariableTok}[1]{\textcolor[rgb]{0.10,0.09,0.49}{{#1}}}
    \newcommand{\ControlFlowTok}[1]{\textcolor[rgb]{0.00,0.44,0.13}{\textbf{{#1}}}}
    \newcommand{\OperatorTok}[1]{\textcolor[rgb]{0.40,0.40,0.40}{{#1}}}
    \newcommand{\BuiltInTok}[1]{{#1}}
    \newcommand{\ExtensionTok}[1]{{#1}}
    \newcommand{\PreprocessorTok}[1]{\textcolor[rgb]{0.74,0.48,0.00}{{#1}}}
    \newcommand{\AttributeTok}[1]{\textcolor[rgb]{0.49,0.56,0.16}{{#1}}}
    \newcommand{\InformationTok}[1]{\textcolor[rgb]{0.38,0.63,0.69}{\textbf{\textit{{#1}}}}}
    \newcommand{\WarningTok}[1]{\textcolor[rgb]{0.38,0.63,0.69}{\textbf{\textit{{#1}}}}}
    
    
    % Define a nice break command that doesn't care if a line doesn't already
    % exist.
    \def\br{\hspace*{\fill} \\* }
    % Math Jax compatability definitions
    \def\gt{>}
    \def\lt{<}
    % Document parameters
    \title{Ron}
    
    
    

    % Pygments definitions
    
\makeatletter
\def\PY@reset{\let\PY@it=\relax \let\PY@bf=\relax%
    \let\PY@ul=\relax \let\PY@tc=\relax%
    \let\PY@bc=\relax \let\PY@ff=\relax}
\def\PY@tok#1{\csname PY@tok@#1\endcsname}
\def\PY@toks#1+{\ifx\relax#1\empty\else%
    \PY@tok{#1}\expandafter\PY@toks\fi}
\def\PY@do#1{\PY@bc{\PY@tc{\PY@ul{%
    \PY@it{\PY@bf{\PY@ff{#1}}}}}}}
\def\PY#1#2{\PY@reset\PY@toks#1+\relax+\PY@do{#2}}

\expandafter\def\csname PY@tok@w\endcsname{\def\PY@tc##1{\textcolor[rgb]{0.73,0.73,0.73}{##1}}}
\expandafter\def\csname PY@tok@c\endcsname{\let\PY@it=\textit\def\PY@tc##1{\textcolor[rgb]{0.25,0.50,0.50}{##1}}}
\expandafter\def\csname PY@tok@cp\endcsname{\def\PY@tc##1{\textcolor[rgb]{0.74,0.48,0.00}{##1}}}
\expandafter\def\csname PY@tok@k\endcsname{\let\PY@bf=\textbf\def\PY@tc##1{\textcolor[rgb]{0.00,0.50,0.00}{##1}}}
\expandafter\def\csname PY@tok@kp\endcsname{\def\PY@tc##1{\textcolor[rgb]{0.00,0.50,0.00}{##1}}}
\expandafter\def\csname PY@tok@kt\endcsname{\def\PY@tc##1{\textcolor[rgb]{0.69,0.00,0.25}{##1}}}
\expandafter\def\csname PY@tok@o\endcsname{\def\PY@tc##1{\textcolor[rgb]{0.40,0.40,0.40}{##1}}}
\expandafter\def\csname PY@tok@ow\endcsname{\let\PY@bf=\textbf\def\PY@tc##1{\textcolor[rgb]{0.67,0.13,1.00}{##1}}}
\expandafter\def\csname PY@tok@nb\endcsname{\def\PY@tc##1{\textcolor[rgb]{0.00,0.50,0.00}{##1}}}
\expandafter\def\csname PY@tok@nf\endcsname{\def\PY@tc##1{\textcolor[rgb]{0.00,0.00,1.00}{##1}}}
\expandafter\def\csname PY@tok@nc\endcsname{\let\PY@bf=\textbf\def\PY@tc##1{\textcolor[rgb]{0.00,0.00,1.00}{##1}}}
\expandafter\def\csname PY@tok@nn\endcsname{\let\PY@bf=\textbf\def\PY@tc##1{\textcolor[rgb]{0.00,0.00,1.00}{##1}}}
\expandafter\def\csname PY@tok@ne\endcsname{\let\PY@bf=\textbf\def\PY@tc##1{\textcolor[rgb]{0.82,0.25,0.23}{##1}}}
\expandafter\def\csname PY@tok@nv\endcsname{\def\PY@tc##1{\textcolor[rgb]{0.10,0.09,0.49}{##1}}}
\expandafter\def\csname PY@tok@no\endcsname{\def\PY@tc##1{\textcolor[rgb]{0.53,0.00,0.00}{##1}}}
\expandafter\def\csname PY@tok@nl\endcsname{\def\PY@tc##1{\textcolor[rgb]{0.63,0.63,0.00}{##1}}}
\expandafter\def\csname PY@tok@ni\endcsname{\let\PY@bf=\textbf\def\PY@tc##1{\textcolor[rgb]{0.60,0.60,0.60}{##1}}}
\expandafter\def\csname PY@tok@na\endcsname{\def\PY@tc##1{\textcolor[rgb]{0.49,0.56,0.16}{##1}}}
\expandafter\def\csname PY@tok@nt\endcsname{\let\PY@bf=\textbf\def\PY@tc##1{\textcolor[rgb]{0.00,0.50,0.00}{##1}}}
\expandafter\def\csname PY@tok@nd\endcsname{\def\PY@tc##1{\textcolor[rgb]{0.67,0.13,1.00}{##1}}}
\expandafter\def\csname PY@tok@s\endcsname{\def\PY@tc##1{\textcolor[rgb]{0.73,0.13,0.13}{##1}}}
\expandafter\def\csname PY@tok@sd\endcsname{\let\PY@it=\textit\def\PY@tc##1{\textcolor[rgb]{0.73,0.13,0.13}{##1}}}
\expandafter\def\csname PY@tok@si\endcsname{\let\PY@bf=\textbf\def\PY@tc##1{\textcolor[rgb]{0.73,0.40,0.53}{##1}}}
\expandafter\def\csname PY@tok@se\endcsname{\let\PY@bf=\textbf\def\PY@tc##1{\textcolor[rgb]{0.73,0.40,0.13}{##1}}}
\expandafter\def\csname PY@tok@sr\endcsname{\def\PY@tc##1{\textcolor[rgb]{0.73,0.40,0.53}{##1}}}
\expandafter\def\csname PY@tok@ss\endcsname{\def\PY@tc##1{\textcolor[rgb]{0.10,0.09,0.49}{##1}}}
\expandafter\def\csname PY@tok@sx\endcsname{\def\PY@tc##1{\textcolor[rgb]{0.00,0.50,0.00}{##1}}}
\expandafter\def\csname PY@tok@m\endcsname{\def\PY@tc##1{\textcolor[rgb]{0.40,0.40,0.40}{##1}}}
\expandafter\def\csname PY@tok@gh\endcsname{\let\PY@bf=\textbf\def\PY@tc##1{\textcolor[rgb]{0.00,0.00,0.50}{##1}}}
\expandafter\def\csname PY@tok@gu\endcsname{\let\PY@bf=\textbf\def\PY@tc##1{\textcolor[rgb]{0.50,0.00,0.50}{##1}}}
\expandafter\def\csname PY@tok@gd\endcsname{\def\PY@tc##1{\textcolor[rgb]{0.63,0.00,0.00}{##1}}}
\expandafter\def\csname PY@tok@gi\endcsname{\def\PY@tc##1{\textcolor[rgb]{0.00,0.63,0.00}{##1}}}
\expandafter\def\csname PY@tok@gr\endcsname{\def\PY@tc##1{\textcolor[rgb]{1.00,0.00,0.00}{##1}}}
\expandafter\def\csname PY@tok@ge\endcsname{\let\PY@it=\textit}
\expandafter\def\csname PY@tok@gs\endcsname{\let\PY@bf=\textbf}
\expandafter\def\csname PY@tok@gp\endcsname{\let\PY@bf=\textbf\def\PY@tc##1{\textcolor[rgb]{0.00,0.00,0.50}{##1}}}
\expandafter\def\csname PY@tok@go\endcsname{\def\PY@tc##1{\textcolor[rgb]{0.53,0.53,0.53}{##1}}}
\expandafter\def\csname PY@tok@gt\endcsname{\def\PY@tc##1{\textcolor[rgb]{0.00,0.27,0.87}{##1}}}
\expandafter\def\csname PY@tok@err\endcsname{\def\PY@bc##1{\setlength{\fboxsep}{0pt}\fcolorbox[rgb]{1.00,0.00,0.00}{1,1,1}{\strut ##1}}}
\expandafter\def\csname PY@tok@kc\endcsname{\let\PY@bf=\textbf\def\PY@tc##1{\textcolor[rgb]{0.00,0.50,0.00}{##1}}}
\expandafter\def\csname PY@tok@kd\endcsname{\let\PY@bf=\textbf\def\PY@tc##1{\textcolor[rgb]{0.00,0.50,0.00}{##1}}}
\expandafter\def\csname PY@tok@kn\endcsname{\let\PY@bf=\textbf\def\PY@tc##1{\textcolor[rgb]{0.00,0.50,0.00}{##1}}}
\expandafter\def\csname PY@tok@kr\endcsname{\let\PY@bf=\textbf\def\PY@tc##1{\textcolor[rgb]{0.00,0.50,0.00}{##1}}}
\expandafter\def\csname PY@tok@bp\endcsname{\def\PY@tc##1{\textcolor[rgb]{0.00,0.50,0.00}{##1}}}
\expandafter\def\csname PY@tok@fm\endcsname{\def\PY@tc##1{\textcolor[rgb]{0.00,0.00,1.00}{##1}}}
\expandafter\def\csname PY@tok@vc\endcsname{\def\PY@tc##1{\textcolor[rgb]{0.10,0.09,0.49}{##1}}}
\expandafter\def\csname PY@tok@vg\endcsname{\def\PY@tc##1{\textcolor[rgb]{0.10,0.09,0.49}{##1}}}
\expandafter\def\csname PY@tok@vi\endcsname{\def\PY@tc##1{\textcolor[rgb]{0.10,0.09,0.49}{##1}}}
\expandafter\def\csname PY@tok@vm\endcsname{\def\PY@tc##1{\textcolor[rgb]{0.10,0.09,0.49}{##1}}}
\expandafter\def\csname PY@tok@sa\endcsname{\def\PY@tc##1{\textcolor[rgb]{0.73,0.13,0.13}{##1}}}
\expandafter\def\csname PY@tok@sb\endcsname{\def\PY@tc##1{\textcolor[rgb]{0.73,0.13,0.13}{##1}}}
\expandafter\def\csname PY@tok@sc\endcsname{\def\PY@tc##1{\textcolor[rgb]{0.73,0.13,0.13}{##1}}}
\expandafter\def\csname PY@tok@dl\endcsname{\def\PY@tc##1{\textcolor[rgb]{0.73,0.13,0.13}{##1}}}
\expandafter\def\csname PY@tok@s2\endcsname{\def\PY@tc##1{\textcolor[rgb]{0.73,0.13,0.13}{##1}}}
\expandafter\def\csname PY@tok@sh\endcsname{\def\PY@tc##1{\textcolor[rgb]{0.73,0.13,0.13}{##1}}}
\expandafter\def\csname PY@tok@s1\endcsname{\def\PY@tc##1{\textcolor[rgb]{0.73,0.13,0.13}{##1}}}
\expandafter\def\csname PY@tok@mb\endcsname{\def\PY@tc##1{\textcolor[rgb]{0.40,0.40,0.40}{##1}}}
\expandafter\def\csname PY@tok@mf\endcsname{\def\PY@tc##1{\textcolor[rgb]{0.40,0.40,0.40}{##1}}}
\expandafter\def\csname PY@tok@mh\endcsname{\def\PY@tc##1{\textcolor[rgb]{0.40,0.40,0.40}{##1}}}
\expandafter\def\csname PY@tok@mi\endcsname{\def\PY@tc##1{\textcolor[rgb]{0.40,0.40,0.40}{##1}}}
\expandafter\def\csname PY@tok@il\endcsname{\def\PY@tc##1{\textcolor[rgb]{0.40,0.40,0.40}{##1}}}
\expandafter\def\csname PY@tok@mo\endcsname{\def\PY@tc##1{\textcolor[rgb]{0.40,0.40,0.40}{##1}}}
\expandafter\def\csname PY@tok@ch\endcsname{\let\PY@it=\textit\def\PY@tc##1{\textcolor[rgb]{0.25,0.50,0.50}{##1}}}
\expandafter\def\csname PY@tok@cm\endcsname{\let\PY@it=\textit\def\PY@tc##1{\textcolor[rgb]{0.25,0.50,0.50}{##1}}}
\expandafter\def\csname PY@tok@cpf\endcsname{\let\PY@it=\textit\def\PY@tc##1{\textcolor[rgb]{0.25,0.50,0.50}{##1}}}
\expandafter\def\csname PY@tok@c1\endcsname{\let\PY@it=\textit\def\PY@tc##1{\textcolor[rgb]{0.25,0.50,0.50}{##1}}}
\expandafter\def\csname PY@tok@cs\endcsname{\let\PY@it=\textit\def\PY@tc##1{\textcolor[rgb]{0.25,0.50,0.50}{##1}}}

\def\PYZbs{\char`\\}
\def\PYZus{\char`\_}
\def\PYZob{\char`\{}
\def\PYZcb{\char`\}}
\def\PYZca{\char`\^}
\def\PYZam{\char`\&}
\def\PYZlt{\char`\<}
\def\PYZgt{\char`\>}
\def\PYZsh{\char`\#}
\def\PYZpc{\char`\%}
\def\PYZdl{\char`\$}
\def\PYZhy{\char`\-}
\def\PYZsq{\char`\'}
\def\PYZdq{\char`\"}
\def\PYZti{\char`\~}
% for compatibility with earlier versions
\def\PYZat{@}
\def\PYZlb{[}
\def\PYZrb{]}
\makeatother


    % Exact colors from NB
    \definecolor{incolor}{rgb}{0.0, 0.0, 0.5}
    \definecolor{outcolor}{rgb}{0.545, 0.0, 0.0}



    
    % Prevent overflowing lines due to hard-to-break entities
    \sloppy 
    % Setup hyperref package
    \hypersetup{
      breaklinks=true,  % so long urls are correctly broken across lines
      colorlinks=true,
      urlcolor=urlcolor,
      linkcolor=linkcolor,
      citecolor=citecolor,
      }
    % Slightly bigger margins than the latex defaults
    
    \geometry{verbose,tmargin=1in,bmargin=1in,lmargin=1in,rmargin=1in}
    
    

    \begin{document}
    
    
    \maketitle
    
    

    
    Ron:

Would you be able to assist me with expressing a bivariate multilevel
mediated model in matrix notation?

The model consists of two mediated models. Four equations in total: two
regression equations (\(y1\) and \(y2\)) and two mediated equations
(\(m1\) and \(m2\)). \(y1\) is radial bone properties, \(y2\) is tibial
bone properties, \(m1\) is grip strength and \(m2\) is knee strength.
\(Z\) and \(W\) are exogenous variables. Grip strength mediates the
effects of \(W\) on radial bone properties and knee strength mediates
the effect of \(W\)\$ on tibial bone properties.

All equations includes include random-individual effects; \(\gamma_i\)
in the regression equations and \(\delta_i\) in the mediated equations.
The random-individual effects are correlated across the regression and
mediated equations (i.e., correlated random effects models). The
regression equation residuals are correlated, and the mediation
equations are corrected (i.e., seemingly unrelated models).

The base specification restricts the coefficients to equal across the
regression equations and mediated equations, respectively. Without
matrices the models can be expressed as:

\[ 
y1_{i,t} = \beta_0 + \beta_1 m1_{i,t} + \Theta Z_{i,t}' \Theta + \gamma_{i} + \epsilon1_{i,t} 
\]

\[
m1_{i,t} = \Omega W_{i,t} + \delta_{i} + \mu1_{i,t} 
\]

\[ 
y2_{i,t} = \beta_0 + \beta_1 m2_{i,t} + \Theta Z_{i,t}' \Theta + \gamma_{i} + \epsilon2_{i,t} 
\]

\[ 
m2_{i,t} = \Omega W_{i,t} + \delta_{i} + \mu2_{i,t} 
\]

\[ 
\Sigma = 
\begin{bmatrix}
 \sigma^2_{\epsilon1} & \sigma_{\epsilon1, \epsilon2}  \\
 \sigma_{\epsilon2, \epsilon1} & \sigma^2_{\epsilon2}  \\
\end{bmatrix} 
\]

\[ 
\Sigma = 
\begin{bmatrix}
 \sigma^2_{\mu1} & \sigma_{\mu1, \mu2}  \\
 \sigma_{\mu2, \mu1} & \sigma^2_{\mu2}  \\
\end{bmatrix} 
\]

\[
\Sigma = 
\begin{bmatrix}
 \sigma^2_{\gamma} & \sigma_{\gamma, \delta}  \\
 \sigma_{\delta, \gamma} & \sigma^2_{\delta}  \\
\end{bmatrix} 
\]

I would like to express the entire model using matrices: \(\mathbf{Y}\)
represents both \(y1_{i,t}\) and \(y2_{i,t}\) and \(\mathbf{M}\)
represents both \(m1_{i,t}\) and \(m2_{i,t}\). Make sense?

    \begin{Verbatim}[commandchars=\\\{\}]
{\color{incolor}In [{\color{incolor}5}]:} \PY{k+kn}{import} \PY{n+nn}{pandas}
        \PY{k+kn}{import} \PY{n+nn}{ipystata}
        \PY{k+kn}{from} \PY{n+nn}{ipystata}\PY{n+nn}{.}\PY{n+nn}{config} \PY{k}{import} \PY{n}{config\PYZus{}stata}
        \PY{n}{config\PYZus{}stata}\PY{p}{(}\PY{l+s+s1}{\PYZsq{}}\PY{l+s+s1}{/usr/local/stata/stata\PYZhy{}mp}\PY{l+s+s1}{\PYZsq{}}\PY{p}{)}
        
        \PY{n}{pandas}\PY{o}{.}\PY{n}{set\PYZus{}option}\PY{p}{(}\PY{l+s+s1}{\PYZsq{}}\PY{l+s+s1}{display.float\PYZus{}format}\PY{l+s+s1}{\PYZsq{}}\PY{p}{,} \PY{k}{lambda} \PY{n}{x}\PY{p}{:} \PY{l+s+s1}{\PYZsq{}}\PY{l+s+si}{\PYZpc{}.2f}\PY{l+s+s1}{\PYZsq{}} \PY{o}{\PYZpc{}} \PY{n}{x}\PY{p}{)}
        \PY{n}{pandas}\PY{o}{.}\PY{n}{set\PYZus{}option}\PY{p}{(}\PY{l+s+s1}{\PYZsq{}}\PY{l+s+s1}{max\PYZus{}columns}\PY{l+s+s1}{\PYZsq{}}\PY{p}{,} \PY{l+m+mi}{200}\PY{p}{)}
        \PY{n}{pandas}\PY{o}{.}\PY{n}{set\PYZus{}option}\PY{p}{(}\PY{l+s+s1}{\PYZsq{}}\PY{l+s+s1}{max\PYZus{}rows}\PY{l+s+s1}{\PYZsq{}}\PY{p}{,} \PY{l+m+mi}{400}\PY{p}{)}
        \PY{n}{pandas}\PY{o}{.}\PY{n}{set\PYZus{}option}\PY{p}{(}\PY{l+s+s1}{\PYZsq{}}\PY{l+s+s1}{max\PYZus{}colwidth}\PY{l+s+s1}{\PYZsq{}}\PY{p}{,} \PY{l+m+mi}{150}\PY{p}{)}
        \PY{n}{pandas}\PY{o}{.}\PY{n}{set\PYZus{}option}\PY{p}{(}\PY{l+s+s1}{\PYZsq{}}\PY{l+s+s1}{mode.sim\PYZus{}interactive}\PY{l+s+s1}{\PYZsq{}}\PY{p}{,} \PY{k+kc}{True}\PY{p}{)}
        \PY{n}{pandas}\PY{o}{.}\PY{n}{set\PYZus{}option}\PY{p}{(}\PY{l+s+s1}{\PYZsq{}}\PY{l+s+s1}{colheader\PYZus{}justify}\PY{l+s+s1}{\PYZsq{}}\PY{p}{,} \PY{l+s+s1}{\PYZsq{}}\PY{l+s+s1}{center}\PY{l+s+s1}{\PYZsq{}}\PY{p}{)}
\end{Verbatim}


    \begin{Verbatim}[commandchars=\\\{\}]
{\color{incolor}In [{\color{incolor}6}]:} \PY{n}{dm} \PY{o}{=} \PY{n}{pandas}\PY{o}{.}\PY{n}{read\PYZus{}csv}\PY{p}{(}\PY{l+s+s1}{\PYZsq{}}\PY{l+s+s1}{out/data\PYZus{}analysis.csv}\PY{l+s+s1}{\PYZsq{}}\PY{p}{)}
        \PY{n}{dm}\PY{o}{.}\PY{n}{head}\PY{p}{(}\PY{p}{)}
\end{Verbatim}


\begin{Verbatim}[commandchars=\\\{\}]
{\color{outcolor}Out[{\color{outcolor}6}]:}    ID   Sequence   calo   height  weight  Godin\_PA  id   season  session  \textbackslash{}
        0  100      1    1572.00  152.00  44.20    37.00    100  spring     1      
        1  100      3    2219.01  160.80  48.80    48.00    100  spring     3      
        2  100      5    2599.83  173.30  63.50    44.00    100  spring     5      
        3  100      7    2482.12  177.40  67.30    66.00    100  spring     7      
        4  101      1    1549.99  158.10  47.30    25.00    101  spring     1      
        
          gender  rsos  tsos  grip  ptiso  biodex  matlab  ntxc   matu   mvh    age  \textbackslash{}
        0   boy   3828  3601   nan 113.93  129.10  113.93 711.85 -1.67 105.71 11.75   
        1   boy   3898  3629 27.00 136.02  138.70  136.02 760.09 -0.71  93.62 12.71   
        2   boy   3851  3677 37.00 177.05  178.00  177.05 543.73  0.41  98.14 13.83   
        3   boy   3952  3740 40.50 205.80  205.80  205.80 454.36  1.17  88.14 14.74   
        4   boy   3682  3603   nan 133.05  139.10  133.05 938.00 -1.63  89.29 11.45   
        
          Exclusion  NoSOS         NoGrip        Noptiso Nontx Nomvh  T    \_o  \_id  \textbackslash{}
        0    none    nan    Not taken sequence 1    NaN    NaN   NaN  NaN   1   1    
        1    none    nan                     NaN    NaN    NaN   NaN  NaN   2   1    
        2    none    nan                     NaN    NaN    NaN   NaN  NaN   3   1    
        3    none    nan                     NaN    NaN    NaN   NaN  NaN   4   1    
        4    none    nan    Not taken sequence 1    NaN    NaN   NaN  NaN   5   2    
        
           \_v  size \_season  omit  boys  girls  trip  st\_rsos  st\_tsos  st\_grip  \textbackslash{}
        0   1  4.00  spring  nan     1     0      1    0.12    -0.79      nan     
        1   2   nan     NaN  nan     1     0      2    0.82    -0.53     0.48     
        2   3   nan     NaN  nan     1     0      3    0.35    -0.10     1.83     
        3   4   nan     NaN  nan     1     0      4    1.36     0.48     2.31     
        4   1  1.00  spring  nan     1     0      1   -1.34    -0.77      nan     
        
           st\_ptiso  st\_biodex  st\_matlab  st\_ntxc  st\_calo  st\_matu  st\_mvh  st\_age  \textbackslash{}
        0   -0.25     -0.04      -0.25      0.65    -0.04    -0.40    -0.03   -0.05    
        1    0.14      0.13       0.14      0.84     1.30     0.07    -0.32    0.44    
        2    0.86      0.81       0.86      0.01     2.10     0.63    -0.21    1.00    
        3    1.37      1.29       1.37     -0.32     1.85     1.00    -0.46    1.46    
        4    0.09      0.13       0.09      1.51    -0.09    -0.38    -0.43   -0.20    
        
           st\_height  st\_weight  st\_Godin\_PA  
        0   -0.01      -0.13       -0.93      
        1    0.64       0.19       -0.65      
        2    1.55       1.20       -0.75      
        3    1.85       1.47       -0.20      
        4    0.44       0.08       -1.23      
\end{Verbatim}
            
    \begin{Verbatim}[commandchars=\\\{\}]
{\color{incolor}In [{\color{incolor}8}]:} \PY{o}{\PYZpc{}\PYZpc{}}\PY{k}{stata} \PYZhy{}\PYZhy{}data dm 
        
        rename biodex biod
        gen lng = ln(grip)
        gen lnb = ln(biod)
        gen lnr = ln(rsos)
        gen lnt = ln(tsos)
        
        rename st\PYZus{}rsos r
        rename st\PYZus{}tsos t
        rename st\PYZus{}grip g
        rename st\PYZus{}biod b
        rename st\PYZus{}matu m
        rename st\PYZus{}ntxc n
        rename st\PYZus{}calo c
        rename st\PYZus{}mvh v
        
        gen rb = boys * r
        gen rg = girls * r
        
        set cformat \PYZpc{}9.4f
        
        \PYZsh{}delimit ;
        
        eststo m1: gsem
            (r \PYZlt{}\PYZhy{} g@k1   \PYZus{}cons@kk)
            (t \PYZlt{}\PYZhy{} b@k1 v \PYZus{}cons@kk)
            (r \PYZlt{}\PYZhy{} m@k2 n@k3  M1[id]@1 \PYZus{}cons@cc) 
            (g b \PYZlt{}\PYZhy{} m@k4 c@k5 M2[id]@1), 
            covstruct(\PYZus{}lexogenous, diagonal)   
            nocapslatent latent(M1 M2) 
            cov(e.r*e.t) cov(e.g*e.b) cov(M1[id]*M2[id])
            nohead nolog 
            ;
\end{Verbatim}


    \begin{Verbatim}[commandchars=\\\{\}]

(43 missing values generated)
delimiter now ;>     (r <- g@k1   \_cons@kk)
>     (t <- b@k1 v \_cons@kk)
>     (r <- m@k2 n@k3  M1[id]@1 \_cons@cc) 
>     (g b <- m@k4 c@k5 M2[id]@1), 
>     covstruct(\_lexogenous, diagonal)   
>     nocapslatent latent(M1 M2) 
>     cov(e.r*e.t) cov(e.g*e.b) cov(M1[id]*M2[id])
>     nohead nolog 
>     ;
 ( 1)  [r]g - [t]b = 0
 ( 2)  [r]M1[id] = 1
 ( 3)  [g]m - [b]m = 0
 ( 4)  [g]c - [b]c = 0
 ( 5)  [g]M2[id] = 1
 ( 6)  [b]M2[id] = 1
-----------------------------------------------------------------------------------
                  |      Coef.   Std. Err.      z    P>|z|     [95\% Conf. Interval]
------------------+----------------------------------------------------------------
r <-              |
                g |     0.3333     0.0490     6.80   0.000       0.2373      0.4293
                m |     0.0913     0.0534     1.71   0.087      -0.0133      0.1960
                n |    -0.1911     0.0444    -4.30   0.000      -0.2781     -0.1041
                  |
           M1[id] |     1.0000  (constrained)
                  |
            \_cons |    -0.0136     0.0612    -0.22   0.825      -0.1334      0.1063
------------------+----------------------------------------------------------------
t <-              |
                b |     0.3333     0.0490     6.80   0.000       0.2373      0.4293
                v |    -0.1075     0.0520    -2.07   0.039      -0.2094     -0.0056
            \_cons |    -0.0825     0.0517    -1.60   0.111      -0.1838      0.0189
------------------+----------------------------------------------------------------
g <-              |
                m |     0.6284     0.0324    19.39   0.000       0.5649      0.6919
                c |     0.0815     0.0270     3.02   0.003       0.0285      0.1344
                  |
           M2[id] |     1.0000  (constrained)
                  |
            \_cons |    -0.0062     0.0540    -0.12   0.908      -0.1121      0.0996
------------------+----------------------------------------------------------------
b <-              |
                m |     0.6284     0.0324    19.39   0.000       0.5649      0.6919
                c |     0.0815     0.0270     3.02   0.003       0.0285      0.1344
                  |
           M2[id] |     1.0000  (constrained)
                  |
            \_cons |     0.0304     0.0503     0.60   0.545      -0.0681      0.1289
------------------+----------------------------------------------------------------
       var(M1[id])|     0.4438     0.0693                        0.3268      0.6028
       var(M2[id])|     0.3586     0.0470                        0.2774      0.4636
------------------+----------------------------------------------------------------
cov(M2[id],M1[id])|    -0.0725     0.0442    -1.64   0.101      -0.1592      0.0143
------------------+----------------------------------------------------------------
          var(e.r)|     0.3175     0.0403                        0.2475      0.4073
          var(e.t)|     0.8050     0.0663                        0.6850      0.9461
          var(e.g)|     0.2834     0.0283                        0.2330      0.3447
          var(e.b)|     0.1591     0.0169                        0.1293      0.1959
------------------+----------------------------------------------------------------
      cov(e.t,e.r)|     0.2557     0.0552     4.63   0.000       0.1475      0.3638
      cov(e.b,e.g)|     0.0379     0.0139     2.72   0.006       0.0106      0.0651
-----------------------------------------------------------------------------------
> 

end of do-file

    \end{Verbatim}

    \begin{Verbatim}[commandchars=\\\{\}]
{\color{incolor}In [{\color{incolor} }]:} 
\end{Verbatim}



    % Add a bibliography block to the postdoc
    
    
    
    \end{document}
